%%%%%%%%%%%%%%%%%%%%%%%%%%%%%%%%%%%%%%%%%%%%%%%%%%%%%%%%%%%%%%%%%%%
% 
% $Id: test.tex,v 1.1.1.1 2002/01/02 19:36:28 phil Exp $
%
% $Log: test.tex,v $
% Revision 1.1.1.1  2002/01/02 19:36:28  phil
% initial import into CVS
%
% Revision 1.1  1996/08/13 19:08:34  phil
% removed counters
%
% Revision 1.11  1996/08/12 23:20:01  phil
% ready for carmen to finish \em
%
% Revision 1.10  1996/08/12 21:38:37  cguo
% change defaults
%
% Revision 1.9  1995/10/29 22:46:50  phil
% trouble with RCS
%
% Revision 1.8  1995/10/29 22:39:09  phil
% fixed rcsfile
%
% Revision 1.7  1995/10/29 22:18:08  phil
% new name
%
% Revision 1.6  1995/10/29 22:02:18  phil
% moved page number to upper right
%
% Revision 1.5  1995/10/16 02:18:15  phil
% added nummargins.sty to make the margins bigger
%
% Revision 1.4  1995/08/14 21:13:56  stockie
% *** empty log message ***
%
% Revision 1.2  1995/08/14  20:28:19  stockie
% - update for new 'defaults' files.
% - add appendix 2 (technical notes)
%
% Revision 1.1  1995/07/18  21:37:29  stockie
% Initial revision
%
%
%%%%%%%%%%%%%%%%%%%%%%%%%%%%%%%%%%%%%%%%%%%%%%%%%%%%%%%%%%%%%%%%%%%

\documentclass{article}

%%%%%%%%%%%%%%%%%%%%%%%%%%%%%%%%%%%%%%%%%%%%%%%%%%%%%%%%%%%%%%%%%%%

\usepackage{html}
\usepackage{/home/numeric/labs/defaults/nummargins}

\newcommand{\labnumber}{1}
\input{/home/numeric/labs/defaults/lab1-defs}
\input{/home/numeric/labs/defaults/lab-defs}

\usepackage{makeidx}
\usepackage{graphicx}   % replaces psfig
\usepackage{color}  

\usepackage{changebar}
\driver{dvips}

%\usepackage{showlabels}
%\usepackage{float}     % for \listof{}{}
%\usepackage{showidx}

%%%%%%%%%%%%%%%%%%%%%%%%%%%%%%%%%%%%%%%%%%%%%%%%%%%%%%%%%%%%%%%%%%%
%  Add RCS heading to each page of the document.
%%%%%%%%%%%%%%%%%%%%%%%%%%%%%%%%%%%%%%%%%%%%%%%%%%%%%%%%%%%%%%%%%%%
\usepackage{rcs,fancyheadings}
\pagestyle{fancy} 
\RCS $RCSfile: test.tex,v $
\RCS $Revision: 1.1.1.1 $
\RCS $Date: 2002/01/02 19:36:28 $
\lhead{\RCSRCSfile} 
\rhead{Lab 1: page~\thepage} 
\chead{\RCSRevision}
\lfoot{}
\cfoot{}
\rfoot{}

%%%%%%%%%%%%%%%%%%%%%%%%%%%%%%%%%%%%%%%%%%%%%%%%%%%%%%%%%%%%%%%%%%%

\title{Laboratory \#1:\\
  An Introduction to the Numerical Solution of
  Differential Equations: Discretization
}
\author{John M. Stockie}
\date{Date printed: \today \\
{\footnotesize [RCS version: \RCSRevision\ $-$\ \RCSDate]}
} 

\makeindex
\makeglossary

%%%%%%%%%%%%%%%%%%%%%%%%%%%%%%%%%%%%%%%%%%%%%%%%%%%%%%%%%%%%%%%%%%%

\begin{document}

%%%%%%%%%%%%%%%%%%%%%%%%%%%%%%%%%%%%%%%%%%%%%%%%%%%%%%%%%%%%%%%%%%%

\newcommand{\ie}{i.e.~}
\newcommand{\eg}{e.g.~}
\newcommand{\dt}{\mbox{$\Delta t$}{}}
\newcommand{\yi}{\mbox{$y_i$}{}}
\newcommand{\ti}{\mbox{$t_i$}{}}
\newcommand{\eqref}[1]{(\ref{#1})}

%%%%%%%%%%%%%%%%%%%%%%%%%%%%%%%%%%%%%%%%%%%%%%%%%%%%%%%%%%%%%%%%%%%

\maketitle
\tableofcontents


\section{Introduction: Why bother with numerical methods?}
\label{lab1:sec:intro}

It will become clear that it is the problems which \emph{do not have
  exact solutions} which are the most interesting or meaningful from a
physical standpoint.  

\begin{latexonly}
\gloss{initial value problem}{a differential equation (or set
of differential equations) along with
  initial values for the unknown functions.  Abbreviated IVP.}
\gloss{IVP}{initial value problem}
\gloss{boundary value problem}{a differential equation (or set
of differential equations) along with
  boundary values for the unknown functions.  Abbreviated BVP.}
\gloss{BVP}{see \emph{boundary value problem}}
\gloss{first order differential equation}{a differential equation
  involving only first derivatives of the unknown functions.}
\gloss{second order differential equation}{a differential equation
  involving only first and second derivatives of the unknown
  functions.}
\end{latexonly}

\begin{example}

  \label{lab1:exm:conduction}

  Consider a small rock, surrounded by air or water, which gains or
  loses heat only by conduction with its surroundings 
  (\ie there are no radiation effects).
  If the rock is small enough, then we can ignore the effects of
  diffusion of heat within the rock, and consider only the flow of heat
  through its surface, where the rock interacts with the surrounding
  medium.  

  It is well known from experimental observations that the 
  rate at which the temperature of the rock changes
  is proportional to 
  the difference between the rock's surface temperature, $T(t)$, 
    and the \emph{ambient temperature}, $T_a$
  (the ambient temperature is
  simply the temperature of the surrounding material, be it air,
  water, \dots).
  This relationship is expressed by the following ordinary
  differential equation

  \begin{equation}
%    \textcolor[named]{Red}{\frac{dT}{dt}} = -\lambda \,
%    \textcolor[named]{Blue}{(T-T_a)} .
    \underbrace{\frac{dT}{dt}}_{\begin{array}{c} 
                                \mbox{\rm rate of change}\\
                                \mbox{\rm of temperature}
                                \end{array}}
    = -\lambda \underbrace{(T-T_a)}_{\begin{array}{c} 
                                \mbox{\rm temperature}\\
                                \mbox{\rm difference}
                                \end{array}} .
    \label{lab1:eq:conduction1d}
  \end{equation}

  and is commonly known as \emph{Newton's Law of Cooling}.
  (The parameter $\lambda$ is defined to be $\lambda = \mu A/cM$, where
  $A$ is the surface area of the rock, 
  $M$ is its mass, $\mu$ its thermal conductivity, and $c$ its
  specific heat.)

  
  If we assume that $\lambda$ is a constant, then the solution to this
  equation is given by 
  \begin{equation}
    T(t) = T_a + (T(0)-T_a)e^{-\lambda t},
    \label{lab1:eq:conduction-soln}
  \end{equation}
  where $T(0)$ is the initial temperature.  

\end{example}




%%%%%%%%%%%%%%%%%%%%%%%%%%%%%%%%%%%%%%%%%%%%%%%%%%%%%%%%%%%%%%%%%%%%%%
\begin{latexonly}
 \newpage
 \section*{Glossary}
  \def\g#1{\goodbreak\vskip0pt plus .5pt\vskip0.5\baselineskip\par\noindent{\bf #1: }\ignorespaces}
  %%%%%%%%%%%%%%%%%%%%%%%%%%%%%%%%%%%%%%%%%%%%%%%%%%%%%%%%%%%%%%%%%%%
%  
% $Id: lab1.tex,v 1.1.1.1 2002/01/02 19:36:28 phil Exp $
%
% $Log: lab1.tex,v $
% Revision 1.1.1.1  2002/01/02 19:36:28  phil
% initial import into CVS
%
% Revision 1.1  2001/12/16 07:16:11  phil
% Initial revision
%
% Revision 1.12  1997/08/29 20:06:07  allen
% *** empty log message ***
%
% Revision 1.11  1996/08/12 23:20:01  phil
% ready for carmen to finish \em
%
% Revision 1.10  1996/08/12 21:38:37  cguo
% change defaults
%
% Revision 1.9  1995/10/29 22:46:50  phil
% trouble with RCS
%
% Revision 1.8  1995/10/29 22:39:09  phil
% fixed rcsfile
%
% Revision 1.7  1995/10/29 22:18:08  phil
% new name
%
% Revision 1.6  1995/10/29 22:02:18  phil
% moved page number to upper right
%
% Revision 1.5  1995/10/16 02:18:15  phil
% added nummargins.sty to make the margins bigger
%
% Revision 1.4  1995/08/14 21:13:56  stockie
% *** empty log message ***
%
% Revision 1.2  1995/08/14  20:28:19  stockie
% - update for new 'defaults' files.
% - add appendix 2 (technical notes)
%
% Revision 1.1  1995/07/18  21:37:29  stockie
% Initial revision
%
%
%%%%%%%%%%%%%%%%%%%%%%%%%%%%%%%%%%%%%%%%%%%%%%%%%%%%%%%%%%%%%%%%%%%

\documentclass{article}

%%%%%%%%%%%%%%%%%%%%%%%%%%%%%%%%%%%%%%%%%%%%%%%%%%%%%%%%%%%%%%%%%%%

%\usepackage{nummargins}
\usepackage{html,geometry}
\geometry{compat2,letterpaper,headsep=5mm,head=10mm,hmargin={20mm,30mm},bottom=10mm,top=5mm}


\newcommand{\labnumber}{1}
\input{../defaults/lab1-defs}
\input{../defaults/lab-defs}

\usepackage{makeidx}
\usepackage{ifpdf}

\ifpdf
    \usepackage[pdftex]{graphicx} 
    \usepackage{hyperref}
    \pdfcompresslevel=0
    \DeclareGraphicsExtensions{.pdf,.jpg,.mps,.png}
\else
    \usepackage{hyperref}
    \usepackage[dvips]{graphicx}
    \DeclareGraphicsRule{.eps.gz}{eps}{.eps.bb}{`gzip -d #1}
    \DeclareGraphicsExtensions{.eps,.eps.gz}
\fi

\usepackage{color}  

\usepackage{changebar}
%\driver{dvips}
\usepackage{natbib}
%\usepackage{showlabels}
%\usepackage{float}     % for \listof{}{}
%\usepackage{showidx}

%%%%%%%%%%%%%%%%%%%%%%%%%%%%%%%%%%%%%%%%%%%%%%%%%%%%%%%%%%%%%%%%%%%
%  Add RCS heading to each page of the document.
%%%%%%%%%%%%%%%%%%%%%%%%%%%%%%%%%%%%%%%%%%%%%%%%%%%%%%%%%%%%%%%%%%%
\usepackage{rcs,fancyheadings}
\pagestyle{fancy} 
\RCS $RCSfile: lab1.tex,v $
\RCS $Revision: 1.1.1.1 $
\RCS $Date: 2002/01/02 19:36:28 $
\lhead{\RCSRCSfile} 
\rhead{Lab 1: page~\thepage} 
\chead{\RCSRevision}
\lfoot{}
\cfoot{}
\rfoot{}

%%%%%%%%%%%%%%%%%%%%%%%%%%%%%%%%%%%%%%%%%%%%%%%%%%%%%%%%%%%%%%%%%%%

\title{Laboratory \#1:\\
  An Introduction to the Numerical Solution of
  Differential Equations: Discretization
}
\author{John M. Stockie}
\date{Date printed: \today \\
{\footnotesize [RCS version: \RCSRevision\ $-$\ \RCSDate]}
} 

\makeindex
\makeglossary

%%%%%%%%%%%%%%%%%%%%%%%%%%%%%%%%%%%%%%%%%%%%%%%%%%%%%%%%%%%%%%%%%%%

\begin{document}

%%%%%%%%%%%%%%%%%%%%%%%%%%%%%%%%%%%%%%%%%%%%%%%%%%%%%%%%%%%%%%%%%%%

\newcommand{\ie}{i.e.~}
\newcommand{\eg}{e.g.~}
\newcommand{\dt}{\mbox{$\Delta t$}{}}
\newcommand{\yi}{\mbox{$y_i$}{}}
\newcommand{\ti}{\mbox{$t_i$}{}}
\newcommand{\eqref}[1]{(\ref{#1})}

%%%%%%%%%%%%%%%%%%%%%%%%%%%%%%%%%%%%%%%%%%%%%%%%%%%%%%%%%%%%%%%%%%%

\maketitle
\tableofcontents

\section*{List of Problems}
\addcontentsline{toc}{section}{List of Problems}

% \begin{latexonly}
% \makeatletter
% %
% \@dottedtocline{1}{1.5em}{2.3em}{Problem~\ref{lab1:prob:osc}:%
%   Weather Balloon}{\pageref{lab1:prob:osc}}
% %
% \makeatother
% \end{latexonly}
% \begin{htmlonly}
% %
% Problem~\ref{lab1:prob:osc}: Weather Balloon
% %
% \end{htmlonly} 


\input sec1
\input sec2
\input sec3
\input sec4
\input sec5
\input sec6
\appendix
\input ap1

\input ap2


\bibliographystyle{jas}
\bibliography{../../refs}

%%%%%%%%%%%%%%%%%%%%%%%%%%%%%%%%%%%%%%%%%%%%%%%%%%%%%%%%%%%%%%%%%%%%%%
\begin{latexonly}
 \newpage
 \section*{Glossary}
  \def\g#1{\goodbreak\vskip0pt plus .5pt\vskip0.5\baselineskip\par\noindent{\bf #1: }\ignorespaces}
  %%%%%%%%%%%%%%%%%%%%%%%%%%%%%%%%%%%%%%%%%%%%%%%%%%%%%%%%%%%%%%%%%%%
%  
% $Id: lab1.tex,v 1.1.1.1 2002/01/02 19:36:28 phil Exp $
%
% $Log: lab1.tex,v $
% Revision 1.1.1.1  2002/01/02 19:36:28  phil
% initial import into CVS
%
% Revision 1.1  2001/12/16 07:16:11  phil
% Initial revision
%
% Revision 1.12  1997/08/29 20:06:07  allen
% *** empty log message ***
%
% Revision 1.11  1996/08/12 23:20:01  phil
% ready for carmen to finish \em
%
% Revision 1.10  1996/08/12 21:38:37  cguo
% change defaults
%
% Revision 1.9  1995/10/29 22:46:50  phil
% trouble with RCS
%
% Revision 1.8  1995/10/29 22:39:09  phil
% fixed rcsfile
%
% Revision 1.7  1995/10/29 22:18:08  phil
% new name
%
% Revision 1.6  1995/10/29 22:02:18  phil
% moved page number to upper right
%
% Revision 1.5  1995/10/16 02:18:15  phil
% added nummargins.sty to make the margins bigger
%
% Revision 1.4  1995/08/14 21:13:56  stockie
% *** empty log message ***
%
% Revision 1.2  1995/08/14  20:28:19  stockie
% - update for new 'defaults' files.
% - add appendix 2 (technical notes)
%
% Revision 1.1  1995/07/18  21:37:29  stockie
% Initial revision
%
%
%%%%%%%%%%%%%%%%%%%%%%%%%%%%%%%%%%%%%%%%%%%%%%%%%%%%%%%%%%%%%%%%%%%

\documentclass{article}

%%%%%%%%%%%%%%%%%%%%%%%%%%%%%%%%%%%%%%%%%%%%%%%%%%%%%%%%%%%%%%%%%%%

%\usepackage{nummargins}
\usepackage{html,geometry}
\geometry{compat2,letterpaper,headsep=5mm,head=10mm,hmargin={20mm,30mm},bottom=10mm,top=5mm}


\newcommand{\labnumber}{1}
\input{../defaults/lab1-defs}
\input{../defaults/lab-defs}

\usepackage{makeidx}
\usepackage{ifpdf}

\ifpdf
    \usepackage[pdftex]{graphicx} 
    \usepackage{hyperref}
    \pdfcompresslevel=0
    \DeclareGraphicsExtensions{.pdf,.jpg,.mps,.png}
\else
    \usepackage{hyperref}
    \usepackage[dvips]{graphicx}
    \DeclareGraphicsRule{.eps.gz}{eps}{.eps.bb}{`gzip -d #1}
    \DeclareGraphicsExtensions{.eps,.eps.gz}
\fi

\usepackage{color}  

\usepackage{changebar}
%\driver{dvips}
\usepackage{natbib}
%\usepackage{showlabels}
%\usepackage{float}     % for \listof{}{}
%\usepackage{showidx}

%%%%%%%%%%%%%%%%%%%%%%%%%%%%%%%%%%%%%%%%%%%%%%%%%%%%%%%%%%%%%%%%%%%
%  Add RCS heading to each page of the document.
%%%%%%%%%%%%%%%%%%%%%%%%%%%%%%%%%%%%%%%%%%%%%%%%%%%%%%%%%%%%%%%%%%%
\usepackage{rcs,fancyheadings}
\pagestyle{fancy} 
\RCS $RCSfile: lab1.tex,v $
\RCS $Revision: 1.1.1.1 $
\RCS $Date: 2002/01/02 19:36:28 $
\lhead{\RCSRCSfile} 
\rhead{Lab 1: page~\thepage} 
\chead{\RCSRevision}
\lfoot{}
\cfoot{}
\rfoot{}

%%%%%%%%%%%%%%%%%%%%%%%%%%%%%%%%%%%%%%%%%%%%%%%%%%%%%%%%%%%%%%%%%%%

\title{Laboratory \#1:\\
  An Introduction to the Numerical Solution of
  Differential Equations: Discretization
}
\author{John M. Stockie}
\date{Date printed: \today \\
{\footnotesize [RCS version: \RCSRevision\ $-$\ \RCSDate]}
} 

\makeindex
\makeglossary

%%%%%%%%%%%%%%%%%%%%%%%%%%%%%%%%%%%%%%%%%%%%%%%%%%%%%%%%%%%%%%%%%%%

\begin{document}

%%%%%%%%%%%%%%%%%%%%%%%%%%%%%%%%%%%%%%%%%%%%%%%%%%%%%%%%%%%%%%%%%%%

\newcommand{\ie}{i.e.~}
\newcommand{\eg}{e.g.~}
\newcommand{\dt}{\mbox{$\Delta t$}{}}
\newcommand{\yi}{\mbox{$y_i$}{}}
\newcommand{\ti}{\mbox{$t_i$}{}}
\newcommand{\eqref}[1]{(\ref{#1})}

%%%%%%%%%%%%%%%%%%%%%%%%%%%%%%%%%%%%%%%%%%%%%%%%%%%%%%%%%%%%%%%%%%%

\maketitle
\tableofcontents

\section*{List of Problems}
\addcontentsline{toc}{section}{List of Problems}

% \begin{latexonly}
% \makeatletter
% %
% \@dottedtocline{1}{1.5em}{2.3em}{Problem~\ref{lab1:prob:osc}:%
%   Weather Balloon}{\pageref{lab1:prob:osc}}
% %
% \makeatother
% \end{latexonly}
% \begin{htmlonly}
% %
% Problem~\ref{lab1:prob:osc}: Weather Balloon
% %
% \end{htmlonly} 


\input sec1
\input sec2
\input sec3
\input sec4
\input sec5
\input sec6
\appendix
\input ap1

\input ap2


\bibliographystyle{jas}
\bibliography{../../refs}

%%%%%%%%%%%%%%%%%%%%%%%%%%%%%%%%%%%%%%%%%%%%%%%%%%%%%%%%%%%%%%%%%%%%%%
\begin{latexonly}
 \newpage
 \section*{Glossary}
  \def\g#1{\goodbreak\vskip0pt plus .5pt\vskip0.5\baselineskip\par\noindent{\bf #1: }\ignorespaces}
  %%%%%%%%%%%%%%%%%%%%%%%%%%%%%%%%%%%%%%%%%%%%%%%%%%%%%%%%%%%%%%%%%%%
%  
% $Id: lab1.tex,v 1.1.1.1 2002/01/02 19:36:28 phil Exp $
%
% $Log: lab1.tex,v $
% Revision 1.1.1.1  2002/01/02 19:36:28  phil
% initial import into CVS
%
% Revision 1.1  2001/12/16 07:16:11  phil
% Initial revision
%
% Revision 1.12  1997/08/29 20:06:07  allen
% *** empty log message ***
%
% Revision 1.11  1996/08/12 23:20:01  phil
% ready for carmen to finish \em
%
% Revision 1.10  1996/08/12 21:38:37  cguo
% change defaults
%
% Revision 1.9  1995/10/29 22:46:50  phil
% trouble with RCS
%
% Revision 1.8  1995/10/29 22:39:09  phil
% fixed rcsfile
%
% Revision 1.7  1995/10/29 22:18:08  phil
% new name
%
% Revision 1.6  1995/10/29 22:02:18  phil
% moved page number to upper right
%
% Revision 1.5  1995/10/16 02:18:15  phil
% added nummargins.sty to make the margins bigger
%
% Revision 1.4  1995/08/14 21:13:56  stockie
% *** empty log message ***
%
% Revision 1.2  1995/08/14  20:28:19  stockie
% - update for new 'defaults' files.
% - add appendix 2 (technical notes)
%
% Revision 1.1  1995/07/18  21:37:29  stockie
% Initial revision
%
%
%%%%%%%%%%%%%%%%%%%%%%%%%%%%%%%%%%%%%%%%%%%%%%%%%%%%%%%%%%%%%%%%%%%

\documentclass{article}

%%%%%%%%%%%%%%%%%%%%%%%%%%%%%%%%%%%%%%%%%%%%%%%%%%%%%%%%%%%%%%%%%%%

%\usepackage{nummargins}
\usepackage{html,geometry}
\geometry{compat2,letterpaper,headsep=5mm,head=10mm,hmargin={20mm,30mm},bottom=10mm,top=5mm}


\newcommand{\labnumber}{1}
\input{../defaults/lab1-defs}
\input{../defaults/lab-defs}

\usepackage{makeidx}
\usepackage{ifpdf}

\ifpdf
    \usepackage[pdftex]{graphicx} 
    \usepackage{hyperref}
    \pdfcompresslevel=0
    \DeclareGraphicsExtensions{.pdf,.jpg,.mps,.png}
\else
    \usepackage{hyperref}
    \usepackage[dvips]{graphicx}
    \DeclareGraphicsRule{.eps.gz}{eps}{.eps.bb}{`gzip -d #1}
    \DeclareGraphicsExtensions{.eps,.eps.gz}
\fi

\usepackage{color}  

\usepackage{changebar}
%\driver{dvips}
\usepackage{natbib}
%\usepackage{showlabels}
%\usepackage{float}     % for \listof{}{}
%\usepackage{showidx}

%%%%%%%%%%%%%%%%%%%%%%%%%%%%%%%%%%%%%%%%%%%%%%%%%%%%%%%%%%%%%%%%%%%
%  Add RCS heading to each page of the document.
%%%%%%%%%%%%%%%%%%%%%%%%%%%%%%%%%%%%%%%%%%%%%%%%%%%%%%%%%%%%%%%%%%%
\usepackage{rcs,fancyheadings}
\pagestyle{fancy} 
\RCS $RCSfile: lab1.tex,v $
\RCS $Revision: 1.1.1.1 $
\RCS $Date: 2002/01/02 19:36:28 $
\lhead{\RCSRCSfile} 
\rhead{Lab 1: page~\thepage} 
\chead{\RCSRevision}
\lfoot{}
\cfoot{}
\rfoot{}

%%%%%%%%%%%%%%%%%%%%%%%%%%%%%%%%%%%%%%%%%%%%%%%%%%%%%%%%%%%%%%%%%%%

\title{Laboratory \#1:\\
  An Introduction to the Numerical Solution of
  Differential Equations: Discretization
}
\author{John M. Stockie}
\date{Date printed: \today \\
{\footnotesize [RCS version: \RCSRevision\ $-$\ \RCSDate]}
} 

\makeindex
\makeglossary

%%%%%%%%%%%%%%%%%%%%%%%%%%%%%%%%%%%%%%%%%%%%%%%%%%%%%%%%%%%%%%%%%%%

\begin{document}

%%%%%%%%%%%%%%%%%%%%%%%%%%%%%%%%%%%%%%%%%%%%%%%%%%%%%%%%%%%%%%%%%%%

\newcommand{\ie}{i.e.~}
\newcommand{\eg}{e.g.~}
\newcommand{\dt}{\mbox{$\Delta t$}{}}
\newcommand{\yi}{\mbox{$y_i$}{}}
\newcommand{\ti}{\mbox{$t_i$}{}}
\newcommand{\eqref}[1]{(\ref{#1})}

%%%%%%%%%%%%%%%%%%%%%%%%%%%%%%%%%%%%%%%%%%%%%%%%%%%%%%%%%%%%%%%%%%%

\maketitle
\tableofcontents

\section*{List of Problems}
\addcontentsline{toc}{section}{List of Problems}

% \begin{latexonly}
% \makeatletter
% %
% \@dottedtocline{1}{1.5em}{2.3em}{Problem~\ref{lab1:prob:osc}:%
%   Weather Balloon}{\pageref{lab1:prob:osc}}
% %
% \makeatother
% \end{latexonly}
% \begin{htmlonly}
% %
% Problem~\ref{lab1:prob:osc}: Weather Balloon
% %
% \end{htmlonly} 


\input sec1
\input sec2
\input sec3
\input sec4
\input sec5
\input sec6
\appendix
\input ap1

\input ap2


\bibliographystyle{jas}
\bibliography{../../refs}

%%%%%%%%%%%%%%%%%%%%%%%%%%%%%%%%%%%%%%%%%%%%%%%%%%%%%%%%%%%%%%%%%%%%%%
\begin{latexonly}
 \newpage
 \section*{Glossary}
  \def\g#1{\goodbreak\vskip0pt plus .5pt\vskip0.5\baselineskip\par\noindent{\bf #1: }\ignorespaces}
  \input{lab1.gll}
\end{latexonly}
%%%%%%%%%%%%%%%%%%%%%%%%%%%%%%%%%%%%%%%%%%%%%%%%%%%%%%%%%%%%%%%%%%%%%%
\printindex
%%%%%%%%%%%%%%%%%%%%%%%%%%%%%%%%%%%%%%%%%%%%%%%%%%%%%%%%%%%%%%%%%%%%%%

\end{document}

\end{latexonly}
%%%%%%%%%%%%%%%%%%%%%%%%%%%%%%%%%%%%%%%%%%%%%%%%%%%%%%%%%%%%%%%%%%%%%%
\printindex
%%%%%%%%%%%%%%%%%%%%%%%%%%%%%%%%%%%%%%%%%%%%%%%%%%%%%%%%%%%%%%%%%%%%%%

\end{document}

\end{latexonly}
%%%%%%%%%%%%%%%%%%%%%%%%%%%%%%%%%%%%%%%%%%%%%%%%%%%%%%%%%%%%%%%%%%%%%%
\printindex
%%%%%%%%%%%%%%%%%%%%%%%%%%%%%%%%%%%%%%%%%%%%%%%%%%%%%%%%%%%%%%%%%%%%%%

\end{document}

\end{latexonly}
%%%%%%%%%%%%%%%%%%%%%%%%%%%%%%%%%%%%%%%%%%%%%%%%%%%%%%%%%%%%%%%%%%%%%%
\printindex
%%%%%%%%%%%%%%%%%%%%%%%%%%%%%%%%%%%%%%%%%%%%%%%%%%%%%%%%%%%%%%%%%%%%%%

\end{document}
