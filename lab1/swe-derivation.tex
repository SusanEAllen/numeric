%%%%%%%%%%%%%%%%%%%%%%%%%%%%%%%%%%%%%%%%%%%%%%%%%%%%%%%%%%%%%%%%%%%%%
\subsection{Shallow Water Equations}
\label{lab1:ap:swe}

I'd like to work the shallow water wave equations into one of my
examples.  For now they're relegated to an appendix :-(

[References: ????]

The example from Susan with the linear
shallow water equations on an $f$-plane over a flat bottom:
\begin{eqnarray}
  \frac{\partial u}{\partial t} - fv = -g \frac{\partial
    \eta}{\partial x}  
  \label{lab1:eq:swe1} \\
  \frac{\partial v}{\partial t} + fu = -g \frac{\partial
    \eta}{\partial y}
  \label{lab1:eq:swe2} \\
  \frac{\partial \eta}{\partial t} + h\frac{\partial u}{\partial x} +
  h\frac{\partial v}{\partial y} = 0 
  \label{lab1:eq:swe3}
\end{eqnarray}
where $\vec{u}=(u,v)$ is the horizontal velocity, $f$ is the
Coriolis frequency, $g$ is the acceleration due to gravity, $\eta$
is the surface elevation, and $h$ is the undisturbed depth of the
fluid.  These three equations form a set of coupled, linear, partial
differential equations.

With some manipulation, these equations can be combined into a
single, linear partial differential equation (problem)
\begin{eqnarray}
  \left( \frac{\partial^2}{\partial t^2} + f^2 \right) \eta - gh
  \left( \frac{\partial^2}{\partial x^2} + \frac{\partial^2}{\partial
      y^2} \right) \eta = 0. \label{lab1:eq:swe4}
\end{eqnarray}

If we assume a wave form in the $x$-direction, 
\[
  \eta(x,y,t) = \eta_0(y) \exp [i(kx-\omega t)], 
\]
then the partial differential equation \eqref{lab1:eq:swe4} becomes
(problem) 
\begin{eqnarray}
  gh \frac{d^2\eta_0}{dy^2} + (\omega^2+ghk^2-f^2)\eta_0 = 0,
  \label{lab1:eq:swe5} 
\end{eqnarray}
which is a second order, linear, ordinary differential equation.

These two equations \eqref{lab1:eq:swe4} and \eqref{eq:swe5} are the two
physical examples that will be used in this laboratory.  These
equations describe the motion of a shallow layer of water in a
constantly rotating frame under small amplitude motions.  One of the
solutions of both these equations are the Poincar\'e waves.  Over a
flat bottom, one can solve these equations analytically; however,
given topographic variation ({\em what changes does this introduce
  in the equations?}), the slightly more complicated equations can
only be solved numerically.

We can either use shallow water equations as prototypes of ODE and
PDE problems, or deal with the simpler variants in the problems
above.  In the first case, I think I need to deal with simpler
equations anyway, in order to illustrate concepts in numerics.
Consequently, I will want to introduce the shallow water equations,
and then also further simplifications which I will then use for
demonstrating the basic numerical concepts, leaving at least some of
the work on the shallow water equations for exercises.

