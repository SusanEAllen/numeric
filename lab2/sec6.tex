%%%%%%%%%%%%%%%%%%%%%%%%%%%%%%%%%%%%%%%%%%%%%%%%%%%%%%%%%%%%%%%%%%%
% 
% $Id: sec6.tex,v 1.1.1.1 2002/01/02 19:36:48 phil Exp $
%
% $Log: sec6.tex,v $
% Revision 1.1.1.1  2002/01/02 19:36:48  phil
% initial import into CVS
%
% Revision 1.3  1997/08/28 16:40:07  cguo
% *** empty log message ***
%
% Revision 1.2  1996/04/29 19:07:06  stockie
% ready for carmen
%
% Revision 1.1  1995/08/29  21:06:41  stockie
% Initial revision
%
% Revision 1.2  1995/06/29  21:26:21  stockie
% *** empty log message ***
%
%
%%%%%%%%%%%%%%%%%%%%%%%%%%%%%%%%%%%%%%%%%%%%%%%%%%%%%%%%%%%%%%%%%%%
\section{Difference Approximations of Higher Derivatives}
\label{lab2:sec:higher-derivs}

Higher derivatives can be discretized in a similar way to what we did
for first derivatives.
Let's consider for now only the second derivative, for which one
possible approximation is the second order centered formula:
\[
  \frac{y(t_{i+1})-2y(t_i)+y(t_{i-1})}{(\dt)^2} = 
  y^{\prime\prime}(t_i) + {\cal O}((\dt)^2),
\]
There are, of course, many other possible formulae that we might use,
but this is the most commonly used.

\begin{problem}
  \label{lab2:prob:taylor-higher}
  \begin {itemize}
  \item a)
  Use Taylor series to derive this formula.  
  \item b)
  Derive a higher order approximation.
  \end {itemize}
\end{problem} 

%%% Local Variables: 
%%% mode: latex
%%% TeX-master: "lab2"
%%% End: 
